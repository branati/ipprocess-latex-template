%
% Portuguese-BR vertion
% 
\documentclass{article}

\usepackage{ipprocess}
% Use longtable if you want big tables to split over multiple pages.
% \usepackage{longtable}
\usepackage[utf8]{inputenc} 
\usepackage[T1]{fontenc}
\pagestyle{fancy}
\usepackage{libertine}

\sloppy

\title{<Título do Projeto>}

\graphicspath{{./pictures/}} % Pictures dir
\makeindex
\begin{document}

\capa{<versão>}{<mês>}{<ano>}{<Título do Projeto>}{<Título do Documento>}{<Instituição>}
\newpage

%%%%%%%%%%%%%%%%%%%%%%%%%%%%%%%%%%%%%%%%%%%%%%%%%%
%% Revision History
%%%%%%%%%%%%%%%%%%%%%%%%%%%%%%%%%%%%%%%%%%%%%%%%%%
\section*{\center Histórico de Revisões}
  \vspace*{1cm}
  \begin{table}[ht]
    \centering
    \begin{tabular}[pos]{|m{2cm} | m{7.2cm} | m{3.8cm}|} 
      \hline
      \cellcolor[gray]{0.9}
      \textbf{Date} & \cellcolor[gray]{0.9}\textbf{Description} & \cellcolor[gray]{0.9}\textbf{Author(s)}\\ \hline
      \hline
      \small xx/xx/xxxx & \small <Descrição> & \small <Autor(es)> \\ \hline      
      \small xx/xx/xxxx &
      \begin{small}
        \begin{itemize}
          \item Exemplo de;
          \item Revisões em lista;
        \end{itemize}
      \end{small} & \small <Autor(es)> \\ \hline 
    \end{tabular}
  \end{table}

\newpage

% TOC instantiation
\tableofcontents
\newpage

%%%%%%%%%%%%%%%%%%%%%%%%%%%%%%%%%%%%%%%%%%%%%%%%%%
%% Document main content
%%%%%%%%%%%%%%%%%%%%%%%%%%%%%%%%%%%%%%%%%%%%%%%%%%
\section{Introdução}

\subsection{Propósito}

\subsection{Escopo}
    
\subsection{Acreônimos e Abreviações}
  
  \FloatBarrier
  \begin{table}[H]
    \begin{center} \label{tab:definitions}
      \begin{tabular}[pos]{|m{2cm} | m{9cm}|} 
        \hline
        \cellcolor[gray]{0.9}\textbf{Acronym} & \cellcolor[gray]{0.9}\textbf{Description} \\ \hline
        RISC & Reduced Instruction Set Computer \\ \hline
        GPR & General Purpose Registers \\ \hline
        FPGA & Field Gate Programmable Array \\ \hline
      \end{tabular}
    \end{center}
  \end{table}
  

\section{Posicionamento}
\subsection{Oportunidade de Negócios}

\subsection{Descrição do Problema}

\subsection{Sentença de Posição do Produto}

\section{Equipe de Projeto}
\subsection{Stakeholders}
  \FloatBarrier
  \begin{table}[H]
    \begin{center} \label{tab:definitions}
      \begin{tabular}[pos]{|m{4cm} | m{5cm}| m{6cm} |} 
        \hline
        \cellcolor[gray]{0.9}\textbf{Nome} & \cellcolor[gray]{0.9}\textbf{Descrição} & \cellcolor[gray]{0.9}\textbf{Responsabilidades}\\ \hline
         & & \\ \hline
       
      \end{tabular}
    \end{center}
  \end{table}
\subsection{Usuários}    

  \FloatBarrier
  \begin{table}[H]
    \begin{center} \label{tab:definitions}
      \begin{tabular}[pos]{|m{6cm} | m{9cm}|} 
        \hline
        \cellcolor[gray]{0.9}\textbf{Nome/Papel/Contato} &\cellcolor[gray]{0.9}\textbf{Responsabilidades}\\ \hline
        <Nome> \newline 
        \textbf{<Papel>} \newline 
        <Contato> & \\ \hline
       
      \end{tabular}
    \end{center}
  \end{table}

\section{Documentos}
  
  \begin{enumerate}
  \item \textbf{Título do documento:} Descrição breve do conteúdo do documento.
  \end{enumerate}

\section{Visão Geral do Produto}
\subsection{Perspectiva do Produto}

\subsection{Características}

\subsection{Licenças}

\subsection{Entrega e Instalação}

\subsection{Dependências e Restrições}

\section{Outros Requisitos do Produto}

\subsection{Padrões Aplicados}

\subsection{Requisitos de Hardware e Software}

% Optional bibliography section
% To use bibliograpy, first provide the ipprocess.bib file on the root folder.
% \bibliographystyle{ieeetr}
% \bibliography{ipprocess}

\end{document}
