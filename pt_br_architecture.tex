%
% Portuguese-BR vertion
% 
\documentclass{report}

\usepackage{ipprocess}
% Use longtable if you want big tables to split over multiple pages.
% \usepackage{longtable}
\usepackage[utf8]{inputenc} 
\usepackage[brazil]{babel} % Uncomment for portuguese

\sloppy

\graphicspath{{./pictures/}} % Pictures dir
\makeindex
\begin{document}

\DocumentTitle{Documento de Arquitetura}
\Project{Título do Projeto}
\Organization{Nome da Instituição}
\Version{Build 2.0a}

\capa
\newpage
\newpage

%%%%%%%%%%%%%%%%%%%%%%%%%%%%%%%%%%%%%%%%%%%%%%%%%%
%% Revision History
%%%%%%%%%%%%%%%%%%%%%%%%%%%%%%%%%%%%%%%%%%%%%%%%%%
\chapter*{Histórico de Revisões}
  \vspace*{1cm}
  \begin{table}[ht]
    \centering
    \begin{tabular}[pos]{|m{2cm} | m{8cm} | m{4cm}|} 
      \hline
      \cellcolor[gray]{0.9}
      \textbf{Date} & \cellcolor[gray]{0.9}\textbf{Descrição} & \cellcolor[gray]{0.9}\textbf{Autor(s)}\\
      \hline
      xx/xx/xxxx &  <Descrição> & <Autor(es)> \\ \hline      
      xx/xx/xxxx &
      \begin{itemize}
        \item Exemplo de;
        \item Revisões em lista;
      \end{itemize}
      & <Autor(es)> \\ \hline 
    \end{tabular}
  \end{table}

% TOC instantiation
\tableofcontents

%%%%%%%%%%%%%%%%%%%%%%%%%%%%%%%%%%%%%%%%%%%%%%%%%%
%% Document main content
%%%%%%%%%%%%%%%%%%%%%%%%%%%%%%%%%%%%%%%%%%%%%%%%%%
\chapter{Introdução}
  
  \section{Propósito do Documento}
  Este documento descreve a arquitetura do projeto \ipPROCESSProject, incluindo especificações do circuitos internos e máquinas de estados de cada componente. Ele também apresenta diagramas de classe, definições de entrada e saída e diagramas de temporização.O principal objetivo deste documento é definir as especificações do projeto \ipPROCESSProject e prover uma visão geral completa do mesmo.

  \noindent \textcolor{red}{\textit{Informações adicionais podem ser incluídas nesta seção. Entretanto, via de regra, ela não deve se extender por muitos parágrafos.}}

  \section{Stakeholders}
    \FloatBarrier
    \begin{table}[H] 
      \begin{center}
        \begin{tabular}[pos]{|m{6cm} | m{8cm}|} 
          \hline 
          \cellcolor[gray]{0.9}\textbf{Nome} & \cellcolor[gray]{0.9}\textbf{Papel/Responsabilidades} \\ \hline
           &  \\ \hline
        \end{tabular}
      \end{center}
    \end{table} 

\section{Visão Geral do Documento}

O presente documento é apresentado como segue:

  \begin{itemize}
   \item \textbf{Capítulo 2 --} Este capítulo apresenta uma visão geral da arquitetura, com foco em entrada e saída do sistema e arquitetura geral do mesmo;
  \end{itemize}


  % inicio das definições do documento
  \section{Definições}
    \FloatBarrier
    \begin{table}[H]
      \begin{center}
        \begin{tabular}[pos]{|m{5cm} | m{9cm}|} 
          \hline
          \cellcolor[gray]{0.9}\textbf{Termo} & \cellcolor[gray]{0.9}\textbf{Descrição} \\ \hline
                          &                       \\ \hline
        \end{tabular}
      \end{center}
    \end{table}  
  % fim

  % inicio da tabela de acronimos e abreviacoes do documento
  \section{Acrônimos e Abreviações}
    \FloatBarrier
    \begin{table}[H]
      \begin{center}
        \begin{tabular}[pos]{|m{2cm} | m{12cm}|} 
          \hline
          \cellcolor[gray]{0.9}\textbf{Sigla} & \cellcolor[gray]{0.9}\textbf{Descrição} \\ \hline
                &   \\ \hline
        \end{tabular}
      \end{center}
    \end{table}  
  % fim

\chapter{Visão Geral da Arquitetura}

  \section{Diagrama de Classe}

  \section{Definições de Entrada e Saída}

  \section{Datapath Interno}

% inicio das descrições de arquitetura para cada componente do sistema
\chapter{Descrição da Arquitetura}

  \section{Nome do Módulo}

    \subsection{Diagrama de Classe}

    \subsection{Definições de Entrada e Saída}

    \subsection{Datapath Interno}

    \subsection{Máquina de Estados}

    \subsection{Temporização}


% Optional bibliography section
% To use bibliograpy, first provide the ipprocess.bib file on the root folder.
% \bibliographystyle{ieeetr}
% \bibliography{ipprocess}

\end{document}
